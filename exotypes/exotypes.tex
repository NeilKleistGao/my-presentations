\documentclass{beamer}

\usepackage{hyperref}

\usetheme{CambridgeUS}
\usecolortheme{crane}

\begin{document}
\title{First-class Runtime Generation of High-performance Types using Exotypes}
\author{Presenter: Cunyuan}
\maketitle

\begin{frame}
	\frametitle{Before We Start}
  \begin{itemize}
  \item This paper is based on the Terra paper.\pause
  \item You can find my previous slide here: \url{https://github.com/NeilKleistGao/my-presentations}
  \end{itemize}
\end{frame}

\begin{frame}
	\frametitle{What's Exotype?}
  \begin{itemize}
  \item So what is Exotype?\pause
  \item exo- $\rightarrow$ outside.\pause
  \item Exotype $\rightarrow$ type from outside(e.g. a database schema).
  \end{itemize}
\end{frame}

\begin{frame}
	\frametitle{An Example}
  \begin{itemize}
  \item Considering reading data from a database.\pause
  \item If we use a static language(e.g. C), we need to define the data structure beforehand.\pause
  \item If we use a dynamic language, we can not control how the data would be stored.\pause
  \item Data may be boxed, stored in a hash table...
  \end{itemize}
\end{frame}

\begin{frame}
	\frametitle{Solution}
  \begin{itemize}
  \item AOP + MSP.\pause
  \end{itemize}
\end{frame}

\begin{frame}
	\frametitle{Reference}
  \begin{itemize}
  \item Zachary DeVito, Daniel Ritchie, Matt Fisher, Alex Aiken, and Pat Hanrahan. 2014. First-class runtime generation of high-performance types using exotypes. In Proceedings of the 35th Conference on Programming Language Design and Implementation. ACM, 11.
  \item dictionary.com. \url{https://www.dictionary.com/browse/exo}
  \item Zachary DeVito, James Hegarty, Alex Aiken, Pat Hanrahan, and Jan Vitek. 2013. Terra: A multi-stage language for highperformance computing. In Proceedings of the 34th ACM SIGPLAN Conference on Programming Language Design and Implementation (PLDI’13). ACM, New York, 105–116. DOI:\url{https://doi.org/10.1145/2491956.2462166}
  \item Terra: A low-level counterpart to Lua. \url{https://terralang.org/}
  \end{itemize}
\end{frame}

\end{document}

