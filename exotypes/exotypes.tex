\documentclass{beamer}

\usepackage{hyperref}
\usepackage{listings}

\lstset{
  columns=fixed,
  frame=none,
  backgroundcolor=\color[RGB]{245,245,244},
  keywordstyle=\color[RGB]{40,40,255},
  numberstyle=\footnotesize\color{darkgray},
  comment=[l]{--},
  commentstyle=\it\color[RGB]{0,96,96},
  stringstyle=\rmfamily\slshape\color[RGB]{128,0,0},
  showstringspaces=false,
  morekeywords={terra, if, else, end, int, then, return, struct, float, local, quote, function, in, do, and, or}
}

\usetheme{CambridgeUS}
\usecolortheme{crane}

\begin{document}
\title{First-class Runtime Generation of High-performance Types using Exotypes}
\author{Presenter: Cunyuan}
\maketitle

\begin{frame}
	\frametitle{Before We Start}
  \begin{itemize}
  \item This paper is based on the Terra paper.\pause
  \item You can find my previous slide here: \url{https://github.com/NeilKleistGao/my-presentations}
  \end{itemize}
\end{frame}

\begin{frame}
	\frametitle{What's Exotype?}
  \begin{itemize}
  \item So what is Exotype?\pause
  \item exo- $\rightarrow$ outside.\pause
  \item Exotype $\rightarrow$ type from outside(e.g. a database schema).
  \end{itemize}
\end{frame}

\begin{frame}
	\frametitle{An Example}
  \begin{itemize}
  \item Considering reading data from a database.\pause
  \item If we use a static language(e.g. C), we need to define the data structure beforehand.\pause
  \item If we use a dynamic language, we can not control how the data would be stored.\pause
  \item Data may be boxed, stored in a hash table...
  \end{itemize}
\end{frame}

\begin{frame}
	\frametitle{Solution}
  \begin{itemize}
  \item MOP + MSP.\pause
  \item MOP: Meta-Object Protocol\pause
  \item In Lua, we use metatables to extend the normal semantics of objects.
  \end{itemize}
\end{frame}

\begin{frame}[fragile]
	\frametitle{Metatables}
  \begin{lstlisting}
  local myobj = {}
  setmetatable(myobj,
  { __index = function(self,field)
    return field end
  })
  print(myobj.somefield)
  \end{lstlisting} \pause

  \begin{itemize}
  \item If the key doesn't exist, the Lua interpreter will call the $\_\_index$ function.\pause
  \item In this case, we can get string $"somefield"$.
  \end{itemize}
\end{frame}

\begin{frame}
	\frametitle{How To Define An Exotype}
  \begin{itemize}
  \item $(() \rightarrow MemoryLayout) \star (OP_1 \rightarrow Quote) \star \dots \star (OP_n \rightarrow Quote)$\pause
  \item Firstly, compute the memory layout.\pause
  \item Given an instance of a primitive operation($OP_i$), generate a concrete expression quote to implement it.
  \end{itemize}
\end{frame}

\begin{frame}[fragile]
	\frametitle{Memory Layout}
  \begin{lstlisting}
    Student = terralib.types.newstruct()
    Student.metamethods.__getentries = function()
    return { {field = "name", type = rawstring },
      {field = "year", type = int} }
    end
  \end{lstlisting}\pause
  \begin{itemize}
  \item We use the metatable $\_\_getentries$.\pause
  \item So we can define the data structure by using external information.\pause
  \item The keyword $struct$ is a syntax sugar.
  \end{itemize}
\end{frame}

\begin{frame}[fragile]
	\frametitle{Memory Layout}
  \begin{lstlisting}
  Student2.metamethods.__getentries = function()
    local file = io.open("data.csv","r")
    --e.g. name,year
    local titles = split(",",file:read("*line"))
    local data = split(",",file:read("*line"))
    local entries = {}
    --loop over entries in titles
    for i,field in ipairs(titles) do
      --is the data a string or an integer?
      local type =
        tonumber(data[i]) and int or rawstring
      entries[i] = { field = field, type = type}
    end
    return entries
  end
  \end{lstlisting}
\end{frame}

\begin{frame}
	\frametitle{Reference}
  \begin{itemize}
  \item Zachary DeVito, Daniel Ritchie, Matt Fisher, Alex Aiken, and Pat Hanrahan. 2014. First-class runtime generation of high-performance types using exotypes. In Proceedings of the 35th ACM SIGPLAN Conference on Programming Language Design and Implementation (PLDI '14). Association for Computing Machinery, New York, NY, USA, 77–88. \url{https://doi.org/10.1145/2594291.2594307}
  \item dictionary.com. \url{https://www.dictionary.com/browse/exo}
  \item Zachary DeVito, James Hegarty, Alex Aiken, Pat Hanrahan, and Jan Vitek. 2013. Terra: A multi-stage language for highperformance computing. In Proceedings of the 34th ACM SIGPLAN Conference on Programming Language Design and Implementation (PLDI’13). ACM, New York, 105–116. DOI:\url{https://doi.org/10.1145/2491956.2462166}
  \item Terra: A low-level counterpart to Lua. \url{https://terralang.org/}
  \end{itemize}
\end{frame}

\end{document}

