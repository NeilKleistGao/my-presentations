\documentclass{beamer}

\usepackage{graphicx}
\usepackage{listings}

\lstset{
  columns=fixed,
  frame=none,
  backgroundcolor=\color[RGB]{245,245,244},
  keywordstyle=\color[RGB]{40,40,255},
  numberstyle=\footnotesize\color{darkgray},
  commentstyle=\it\color[RGB]{0,96,96},
  stringstyle=\rmfamily\slshape\color[RGB]{128,0,0},
  showstringspaces=false,
  morekeywords={terra, if, else, end, int, then, return, struct, float, local, quote, function}
}

\usetheme{AnnArbor}
\usecolortheme{beaver}

\begin{document}
\title{Terra: A Multi-Stage Language for High-Performance Computing}
\author{Zachary DeVito\inst{1} James Hegarty\inst{1} Alex Aiken\inst{1} Pat Hanrahan\inst{1} Jan Vitek\inst{2}}
\institute{\inst{1} Stanford University \inst{2} Purdue University}

\maketitle

\begin{frame}
	\frametitle{Goals}
  \begin{itemize}
  \item Performance matters!\pause
  \item Low-level languages(e.g. C) are good: we need to make best use of features of the target architecture(e.g. vector instructions).\pause
  \item Programming is difficult!\pause
  \item Solution: use high-level languages to generate low-level languages code(e.g. FFTW: OCaml $\rightarrow$ C).
  \end{itemize}
\end{frame}

\begin{frame}
	\frametitle{New Problems}
  \begin{itemize}
  \item In this case, we get three components.\pause
  \item Optimizer: generate plan to guide how to generate code.\pause
  \item Compiler: generate taget code based on the plan.\pause
  \item Runtime: support the generated code and provide feedback to the optimizer.\pause
  \item Problem1: How can we get the runtime statistics in the compiler and generate high-performance code dynamically?\pause
  \item Problem2: How can we re-use legacy libraries?
  \end{itemize}
\end{frame}

\begin{frame}
  \frametitle{Two-Language Design}
  \begin{itemize}
  \item Lua: high-level, dynamically typed, automatic mm, first class functions.\pause
  \item Terra(new!): statically typed, manumal mm.\pause
  \item Use Lua to manipulate Terra code.\pause
  \item Shared lexical scoping, which is hygienic.\pause
  \item Terra code runs independently, to avoid including high-level features.\pause
  \item Lua's stack-based C API makes it easy to interface with legacy code.
  \end{itemize}
\end{frame}

\begin{frame}
	\frametitle{Two-Language Design}
  \includegraphics[scale=0.3]{terra.png}
\end{frame}

\begin{frame}[fragile]
	\frametitle{Some Code Examples}
  \begin{lstlisting}
    terra min(a: int, b: int): int
      if a < b then return a
      else return b end
    end
    struct GreyScaleImage {
      data: &float
      N: int
    }
  \end{lstlisting}
\end{frame}

\begin{frame}
	\frametitle{Features}
  \begin{itemize}
  \item Terra entities are all first-class Lua values.\pause
  \item Terra functions will be executed in LLVM JIT.\pause
  \item You can dump Terra functions to an object file(i.e. something.o in Linux) if you like.\pause
  \item Quotation: using brackets($[]$) for escaping and backtick(expressions)/quote keyword(statements) for creating quotation.
  \end{itemize}
\end{frame}

\begin{frame}[fragile]
  \frametitle{Quotation Example}
  \begin{lstlisting}
    local a = 5
    terra sin5()
      return [ math.sin(a) ]
      end
    function addtwo(a,b)
      return `a + b
    end
    local printtwice = quote
      C.printf("hello\n")
      C.printf("hello\n")
    end
  \end{lstlisting}
\end{frame}

\begin{frame}
	\frametitle{It Just Works!}
  \includegraphics[scale=0.45]{terra2.png}
\end{frame}

\begin{frame}
	\frametitle{It Just Works!}
  \begin{itemize}
  \item Now we can generate code dynamically.\pause
  \item e.g. block the loop nests to make the memory access more friendly to the cache.
  \end{itemize}
\end{frame}

\begin{frame}
	\frametitle{The Formal Calculus: Terra Core}
  \begin{itemize}
  \item For simplicity, Lua := imperative language + first-class functions, Terra := purely functional language\pause
  \item Lua expression: $e$, evaluation of Lua: $\xrightarrow[]{L}$\pause
  \item Terra expression: $\dot{e}$, specialization of Terra: $\xrightarrow[]{S}$\pause
  \item Specialized Terra expression: $\underline{\dot{e}}$, execution of specailized Terra expression: $\xrightarrow[]{T}$
  \end{itemize}
\end{frame}

\begin{frame}
	\frametitle{Terra Core}
  Lua Syntax:
  \newline
  \begin{equation}
    \begin{split}
      e\, ::=\, & b\, |\, \dot{T}\, |\, x\, |\, let\, x\, =\, e\, in\, e\, |\, x\, :=\, e\, |\, e(e)\, |\, fun(x)\{e\}\, |\, tdecl\, | \\ & ter\, e(x:\, e):\, e \{\dot{e}\}\, |\, \backprime \dot{e}\\
      v\, ::=\, & b\, |\, l\, |\, \dot{T}\, |\, <\Gamma, x, e>\, |\, \underline{\dot{e}}\\
      \dot{T} ::=\, & \dot{B}\, |\, \dot{T} \rightarrow \dot{T}
    \end{split} \notag
  \end{equation}
\end{frame}

\begin{frame}
	\frametitle{Terra Core}
  Terra Syntax:
  \newline
  \begin{equation}
    \begin{split}
      \dot{e}\, ::=\, & b\, |\, x\, | \dot{e}(\dot{e}) |\, tlet\, x:\, e\, =\, \dot{e}\, in\, \dot{e}\, |\, [e]\\
      \underline{\dot{e}}\, ::=\, & b\, |\, \underline{\dot{x}}\, | \underline{\dot{e}}(\underline{\dot{e}}) |\, tlet\, \underline{\dot{x}}:\, \dot{T}\, =\, \underline{\dot{e}}\, in\, \underline{\dot{e}}\, |\, l
    \end{split} \notag
  \end{equation}
\end{frame}

\begin{frame}
  \frametitle{Terra Core}
  \begin{equation}
    v\, \Sigma \xrightarrow[]{L} v\, \Sigma \tag{LVAL}
  \end{equation}
  \newline
  \begin{equation}
    \frac{\Sigma = \Gamma, S, F}{x\, \Sigma \xrightarrow[]{L} S(\Gamma{(x)})\, \Sigma}\tag{LVAR}
  \end{equation}
  \newline
  \begin{equation}
    \frac{e_1 \, \Sigma_1 \xrightarrow[]{L} v_1 \, \Sigma_2 \enspace \Sigma_2 = \Gamma, S, F \enspace e_2 \Sigma_2[x \leftarrow v_1] \xrightarrow[]{L} v_2 \Sigma_3}{let \, x = e_1 \, in \, e_2 \, \Sigma \xrightarrow[]{L} v_2(\Sigma_3 \leftarrow \Gamma)} \tag{LLET}
  \end{equation}
  \newline
  \begin{equation}
    \frac{e \, \Sigma \xrightarrow[]{L} v\, \Gamma, S, F \enspace \Gamma{(x)} = a}{x := e \, \Sigma \xrightarrow[]{L} v \, \Gamma, S[a \leftarrow v], F} \tag{LASN}
  \end{equation}
\end{frame}

\begin{frame}
  \frametitle{Terra Core}
	\begin{equation}
    \frac{\Sigma = \Gamma, S, F}{fun(x)\{e\} \, \Sigma \xrightarrow[]{L} <\Gamma, x, e> \, \Sigma} \tag{LFUN}
  \end{equation}
  \newline
  \begin{equation}
    \begin{split}
      e_1 \, \Sigma_1 \xrightarrow[]{L} <\Gamma_1, x, e_3> \enspace e_2 \, \Sigma_2 \xrightarrow[]{L} v_1 \, \Gamma_2, S, F \\
      \frac{a \, fresh \enspace e_3 \, \Gamma_1[x \leftarrow a], S[a \leftarrow v_1], F \xrightarrow[]{L} v_2 \, \Sigma_3}{e_1(e_2) \, \Sigma_1 \xrightarrow[]{L} v_2 \, (\Sigma_3 \leftarrow \Gamma_2)}
    \end{split} \tag{LAPP}
  \end{equation}
  \newline
  \begin{equation}
    \frac{l\, fresh \enspace \Sigma = \Gamma, S, F}{tdecl \Sigma \xrightarrow[]{L} l \, \Gamma, S, F[l \leftarrow \bullet]} \tag{LTDECL}
  \end{equation}
\end{frame}

\begin{frame}
  \frametitle{Terra Core}
	\begin{equation}
    \begin{split}
      & e_1 \, \Sigma_1 \xrightarrow[]{L} l \, \Sigma_2 \enspace e_2 \, \Sigma_2 \xrightarrow[]{L} \dot{T_1} \, \Sigma_3 \enspace e_3 \, \Sigma_3 \xrightarrow[]{L} \dot{T_2} \, \Sigma_4 \\
      & \Sigma_4 = \Gamma_1, S_1, F_1 \enspace \underline{\dot{x}} \, fresh \\
      & \frac{\dot{e} \, \Sigma_4[x \leftarrow \underline{\dot{x}}] \xrightarrow[]{S} \underline{\dot{e}} \, \Gamma_2, S_2, F_2 \enspace F_2(l) = \bullet}{ter \, e_1(x: e_2): e_3\{\dot{e}\} \, \Sigma_1 \xrightarrow[]{L} l \, \Gamma_1, S_2, F_2[l \leftarrow <\underline{\dot{x}}, \dot{T_1}, \dot{T_2}, \underline{\dot{e}}>]}
    \end{split} \tag{LTDEFN}
  \end{equation}
  \newline
  \begin{equation}
    \frac{\dot{e} \, \Sigma_1 \xrightarrow[]{S} \underline{\dot{e}} \, \Sigma_2}{\backprime{\dot{e}} \, \Sigma_1 \xrightarrow[]{L} \underline{\dot{e}} \, \Sigma_2}\tag{LTQUOTE}
  \end{equation}
\end{frame}

\begin{frame}
  \frametitle{Terra Core}
  \begin{equation}
    \begin{split}
      & e_1 \, \Sigma_1 \xrightarrow[]{L} l \, \Sigma_2 \enspace e_2 \, \Sigma_2 \xrightarrow[]{L} b_1 \, \Sigma_3 \\
      & \Sigma_3 = \Gamma, S, F \enspace F(l) = <\underline{\dot{x}}, \dot{T_1}, \dot{T_2}, \underline{\dot{e}}> \enspace b_1 \in \dot{T_1} \\
      & \frac{[\underline{\dot{x}}: \dot{T_1}], [l: \dot{T_1} \rightarrow \dot{T_2}], F_2 \vdash \underline{\dot{e}}: \dot{T_2} \enspace \underline{\dot{e}}[\underline{\dot{x}} \leftarrow b], F \xrightarrow[]{T} b_2}{e_1(e_2) \, \Sigma_1 \xrightarrow[]{L} b_2 \, \Sigma_3}
    \end{split} \tag{LTAPP}
  \end{equation}
\end{frame}

\begin{frame}
	\frametitle{Summary}
  \begin{itemize}
  \item Two-Languages design: Lua + Terra.\pause
  \item Shared lexical scoping.\pause
  \item Seperate Evaluationn: Lua(LuaJIT), Terra(LLVM JIT).\pause
  \item Type Reflection.
  \end{itemize}
\end{frame}

\begin{frame}
	\frametitle{Advantages}
  \begin{itemize}
  \item No need to write C, but the performance is still good.\pause
  \item Generating code dynamically allows us to use runtime information from LuaJIT.\pause
  \item Easy to re-use C/C++ libraries in Lua.\pause
  \end{itemize}
\end{frame}

\begin{frame}
	\frametitle{Shortages}
  \begin{itemize}
  \item Lua is not statically typed.
  \end{itemize}
\end{frame}

\end{document}

